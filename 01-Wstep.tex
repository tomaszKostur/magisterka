\chapter{Wstęp}
\label{sec:wstep}
\textit{Autor: Mateusz Kowalski}

%===============================================================================
\section{Zawartość wstępu}
\label{sec:wstep:co}

\subsection{Opis zagadnienia poruszonego w pracy}
\sloppy
Pierwszy rozdział pracy powinien zawierać opis zagadnienia poruszonego w pracy.

\subsection{Rys historyczny i aktualne zastosowania}
\label{sec:wstep:rys}

Można również opisać historię badañ danego tematu, oraz jego aktualnych zastosowañ.

\subsection{Autorstwo pracy}
\textit{Autor: Rafał Kabaciñski}\\ \\
W przypadku pracy zespołowej pod tytułem każdego rozdziału powinno znaleźć się imię i nazwisko autora danego rozdziału. Autorem rozdziału może być tylko jedna osoba, ale podrozdziały mogą mieć innego autora niż cały rozdział. 

%===============================================================================
\section{Cel pracy}
\label{sec:wstep:cel}

	Wstęp powienien zawierać również rozdział opisujący cele projektu opisywanego w pracy jak na przykład:
	
\begin{itemize}
	\item zapoznanie się z informacjami na temat istniejących rozwiązañ,
	\item stworzenie funkcjonalnego urządzenia,
	\item sprawdzenie wybranych rozwiązañ konstrukcyjnych.
\end{itemize}

\section{Zawartość pracy}
\label{sec:wstep:zawartosc}
	
	Ostatecznie wstęp musi zawierać przewodnik opisujący co znajduje się w dalszych rozdziałach pracy. Na przykład:
	Ogólny zarys stanu wiedzy na temat przygotowania i składu prac dyplomowych przedstawiono w rozdziale drugim. W rozdziale trzecim wprowadzono podstawowe informacje na temat korzystania z tego formatu pracy. Rozdział czwarty stanowi podsumowanie pracy.
